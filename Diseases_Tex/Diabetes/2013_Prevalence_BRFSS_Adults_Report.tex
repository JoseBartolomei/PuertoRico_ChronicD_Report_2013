
%----------------------------------------------------------------------------------------
%            RESULTS
%----------------------------------------------------------------------------------------
\subsubsection{Overall}

%----------------------------FIRST FIGURE------------------------------------------------

% \subsection*{Overall Prevalence}


%latex.default(object = Overall.tabl[, c(-1, -2)], title = "Group",     file = "", rowname = c("Adults"), colhead = c("Prevalence",         "Cases", "Sample Size"), append = TRUE, table.env = T,     here = T, caption = paste(set_Measure, " among adults in ",         set_Country, ", ", set_Years, sep = ""), label = paste("tab:Overall.tabl",         set_DiseaseF, set_Years, sep = "."))%
\begin{table}[H]
\caption{Diabetes prevalence among adults in Puerto Rico, 2013\label{tab:Overall.tabl.Diabetes.2013}} 
\begin{center}
\begin{tabular}{llll}
\hline\hline
\multicolumn{1}{l}{Group}&\multicolumn{1}{c}{Prevalence}&\multicolumn{1}{c}{Cases}&\multicolumn{1}{c}{Sample Size}\tabularnewline
\hline
Adults&14.88 (13.81-15.95)&418,229&6,011\tabularnewline
\hline
\end{tabular}\end{center}

\end{table}



%----------------------------------------------------------------------------------------

\begin{itemize}


\item The Diabetes prevalence for the year 2013 among adults in Puerto Rico was 14.8 (13.8-15.9) percent, 
Table \ref{tab:Overall.tabl.Diabetes.2013}.

\end{itemize}


%%%%%%%%%%%%%%%%%%%%%%%%%%%%%%%%%%%%%%%%%%%%%%%%%%%%%%%%%%%%%%%%%%%%%%%%%%%%%%%%%%%%%%%%%%%%%%%%%%%%
%%%%%%%%%%%%%%%%%%%%%%%%%%%%%%%%%%%%%%%%%%%%%%%%%%%%%%%%%%%%%%%%%%%%%%%%%%%%%%%%%%%%%%%%%%%%%%%%%%%%
%%%%%%%%%%%%                     Definition of matrices for the tables                  %%%%%%%%%%%%
%%%%%%%%%%%%%%%%%%%%%%%%%%%%%%%%%%%%%%%%%%%%%%%%%%%%%%%%%%%%%%%%%%%%%%%%%%%%%%%%%%%%%%%%%%%%%%%%%%%%
%%%%%%%%%%%%%%%%%%%%%%%%%%%%%%%%%%%%%%%%%%%%%%%%%%%%%%%%%%%%%%%%%%%%%%%%%%%%%%%%%%%%%%%%%%%%%%%%%%%%

\newpage
\subsubsection{Socio-demographics}

\begin{itemize}

\item Figure \ref{fig:age.Diabetes.2013} shows that the diabetes prevalence for the age group of 65+ was significantlly higher when compared with the other groups.

\item Adults in the age group of 35-44, 45-54, 55-64, and 65+ had 5.94 times more possibility, 9.49 times more possibility, 14.9 times more possibility, and 20.3 times more possibility, respectively, of reporting diabetes prevalence when compared with the 18-24 group. Those differences were significant (p-value $<$ 0.05). See Table \ref{tab:age.Diabetes.2013}.


\end{itemize}


\begin{figure}[H]
\caption{Diabetes prevalence among adults by age group, 
2013}
\begin{knitrout}
\definecolor{shadecolor}{rgb}{0.969, 0.969, 0.969}\color{fgcolor}

{\centering \includegraphics[width=\maxwidth]{/media/truecrypt2/ORP2/BRFSS/Objects/Country/Puerto_Rico/Disease/Diabetes/Prevalence/Adult/2013ageg2-1} 

}



\end{knitrout}
\label{fig:age.Diabetes.2013}
\end{figure}

%latex.default(tabl.ageg2, title = "Variables", file = "", append = TRUE,     rgroup = "Age group", colhead = c("Prevalence", "Cases (N)",         "OR", "OR(SE)", "p-value"), longtable = F, table.env = T,     here = T, caption = paste(set_Measure, " among adults by age group,",         set_Years, sep = " "), label = paste("tab:age", set_DiseaseF,         set_Years, sep = "."))%
\begin{table}[H]
\caption{Diabetes prevalence  among adults by age group, 2013\label{tab:age.Diabetes.2013}} 
\begin{center}
\begin{tabular}{llllll}
\hline\hline
\multicolumn{1}{l}{Variables}&\multicolumn{1}{c}{Prevalence}&\multicolumn{1}{c}{Cases (N)}&\multicolumn{1}{c}{OR}&\multicolumn{1}{c}{OR(SE)}&\multicolumn{1}{c}{p-value}\tabularnewline
\hline
{\bfseries Age group}&&&&&\tabularnewline
~~18-24&0.98 (0.10-1.86)&  3,710&1.00&0.00&1.00\tabularnewline
~~25-34&2.47 (0.99-3.96)& 12,056&1.69&0.55&0.34\tabularnewline
~~35-44&8.36 (5.63-11.0)& 40,322&5.94&0.50&0.00\tabularnewline
~~45-54&13.5 (10.8-16.2)& 65,561&9.49&0.49&0.00\tabularnewline
~~55-64&23.5 (20.4-26.6)&102,585&14.9&0.49&0.00\tabularnewline
~~65+&35.5 (32.8-38.1)&193,995&20.3&0.49&0.00\tabularnewline
\hline
\end{tabular}\end{center}

\end{table}


\newpage
\begin{itemize}

\item Figure \ref{fig:sex.Diabetes.2013} shows that males had a non-significant \ifthenelse{
1396 < 1568}{lower}{higher}
diabetes prevalence when compared with females.

%\item Table \ref{tab:sex} shows the estimates for persons living with set_Disease in 
%set_Country with it 95\% confidence interval.

\item Among adults, females had 3\% less possibility of reporting diabetes prevalence when compared with males. This difference was not significant (p-value $>$ 0.05). See Table \ref{tab:sex.Diabetes.2013}.

\end{itemize}

\begin{figure}[H]
\caption{Diabetes prevalence among adults by sex group, 
2013}
\begin{knitrout}
\definecolor{shadecolor}{rgb}{0.969, 0.969, 0.969}\color{fgcolor}

{\centering \includegraphics[width=\maxwidth]{/media/truecrypt2/ORP2/BRFSS/Objects/Country/Puerto_Rico/Disease/Diabetes/Prevalence/Adult/2013sex-1} 

}



\end{knitrout}
\label{fig:sex.Diabetes.2013}
\end{figure}

%latex.default(tabl.sex, title = "Variables", file = "", append = TRUE,     rgroup = "Sex group", colhead = c("Prevalence", "Cases (N)",         "OR", "OR(SE)", "p-value"), longtable = F, table.env = T,     here = T, caption = paste(set_Measure, " among adults by sex group,",         set_Years, sep = " "), label = paste("tab:sex", set_DiseaseF,         set_Years, sep = "."))%
\begin{table}[H]
\caption{Diabetes prevalence  among adults by sex group, 2013\label{tab:sex.Diabetes.2013}} 
\begin{center}
\begin{tabular}{llllll}
\hline\hline
\multicolumn{1}{l}{Variables}&\multicolumn{1}{c}{Prevalence}&\multicolumn{1}{c}{Cases (N)}&\multicolumn{1}{c}{OR}&\multicolumn{1}{c}{OR(SE)}&\multicolumn{1}{c}{p-value}\tabularnewline
\hline
{\bfseries Sex group}&&&&&\tabularnewline
~~Males&13.9 (12.3-15.6)&183,567&1.00&0.00&1.00\tabularnewline
~~Females&15.6 (14.3-17.0)&234,661&0.97&0.11&0.82\tabularnewline
\hline
\end{tabular}\end{center}

\end{table}


%###################### educag ####################
\newpage
\begin{itemize}

\item When comparing by education level, those with
some high school
have a significant highest diabetes prevalence when compare with the other education groups. The diabetes prevalence among the Some High School groups was 25 percent.
(Figure \ref{fig:edu.Diabetes.2013}).

\item 
When observing the adjusted odds ratio the group University Graduate had 37\% less possibility of reporting diabetes prevalence than the comparison group (Some High School).
This difference was significant (p-value $<$ 0.05).  Data shown in Table \ref{tab:edu.Diabetes.2013}.

\end{itemize}

\begin{figure}[H]
\caption{Diabetes prevalence among adults by education levels, 
         2013}
\begin{knitrout}
\definecolor{shadecolor}{rgb}{0.969, 0.969, 0.969}\color{fgcolor}

{\centering \includegraphics[width=\maxwidth]{/media/truecrypt2/ORP2/BRFSS/Objects/Country/Puerto_Rico/Disease/Diabetes/Prevalence/Adult/2013educag-1} 

}



\end{knitrout}
 \label{fig:edu.Diabetes.2013}
\end{figure}

%latex.default(tabl.educag3, title = "Variables", file = "", append = TRUE,     rgroup = "Education group", colhead = c("Prevalence", "Cases (N)",         "OR", "OR(SE)", "p-value"), longtable = F, table.env = T,     here = T, caption = paste(set_Measure, " among adults by education levels,",         set_Years, sep = " "), label = paste("tab:edu", set_DiseaseF,         set_Years, sep = "."))%
\begin{table}[H]
\caption{Diabetes prevalence  among adults by education levels, 2013\label{tab:edu.Diabetes.2013}} 
\begin{center}
\begin{tabular}{llllll}
\hline\hline
\multicolumn{1}{l}{Variables}&\multicolumn{1}{c}{Prevalence}&\multicolumn{1}{c}{Cases (N)}&\multicolumn{1}{c}{OR}&\multicolumn{1}{c}{OR(SE)}&\multicolumn{1}{c}{p-value}\tabularnewline
\hline
{\bfseries Education group}&&&&&\tabularnewline
~~Some High School&25.4 (22.7-28.2)&202,434&1.00&0.00&1.00\tabularnewline
~~High School Graduate&14.5 (12.5-16.6)&105,865&0.94&0.14&0.67\tabularnewline
~~Some University&8.92 (7.38-10.4)& 63,386&0.76&0.15&0.08\tabularnewline
~~University Graduate&8.02 (6.73-9.31)& 45,863&0.63&0.17&0.01\tabularnewline
\hline
\end{tabular}\end{center}

\end{table}

 
 
 \newpage
\begin{itemize}

\item Figure \ref{fig:income.Diabetes.2013} shows that adults whose annual household income is 
$<$15k, have a diabetes prevalence higher than the other income groups.

\item Those in the range of 50+k had 34\% less possibility of reporting diabetes than those whose annual income is less than \$ 14,999. This difference was not significant (p-value $>$ 0.05).  For further information, refer to Table \ref{tab:income.Diabetes.2013}.

\end{itemize}

\begin{figure}[H]
\caption{Diabetes prevalence among adults by household income levels, 
         2013}
\begin{knitrout}
\definecolor{shadecolor}{rgb}{0.969, 0.969, 0.969}\color{fgcolor}

{\centering \includegraphics[width=\maxwidth]{/media/truecrypt2/ORP2/BRFSS/Objects/Country/Puerto_Rico/Disease/Diabetes/Prevalence/Adult/2013incomg-1} 

}



\end{knitrout}
 \label{fig:income.Diabetes.2013}
\end{figure}

%latex.default(tabl.incomg, title = "Variables", file = "", append = TRUE,     rgroup = "Income group", colhead = c("Prevalence", "Cases (N)",         "OR", "OR(SE)", "p-value"), longtable = F, table.env = T,     here = T, caption = paste(set_Measure, " among adults by income levels,",         set_Years, sep = " "), label = paste("tab:income", set_DiseaseF,         set_Years, sep = "."))%
\begin{table}[H]
\caption{Diabetes prevalence  among adults by income levels, 2013\label{tab:income.Diabetes.2013}} 
\begin{center}
\begin{tabular}{llllll}
\hline\hline
\multicolumn{1}{l}{Variables}&\multicolumn{1}{c}{Prevalence}&\multicolumn{1}{c}{Cases (N)}&\multicolumn{1}{c}{OR}&\multicolumn{1}{c}{OR(SE)}&\multicolumn{1}{c}{p-value}\tabularnewline
\hline
{\bfseries Income group}&&&&&\tabularnewline
~~\textless15k&18.5 (16.7-20.4)&223,365&1.00&0.00&1.00\tabularnewline
~~15k-\textless25k&12.1 (10.0-14.2)& 78,437&0.84&0.13&0.23\tabularnewline
~~25k-\textless35k&11.3 (8.44-14.3)& 22,934&0.90&0.19&0.60\tabularnewline
~~35k-\textless50k&8.63 (5.41-11.8)& 12,497&0.87&0.26&0.61\tabularnewline
~~50+k&6.28 (4.10-8.46)&  8,948&0.66&0.23&0.08\tabularnewline
\hline
\end{tabular}\end{center}

\end{table}

%%%%%%%%%%%%%%%%% Marial Status %%%%%%%%%%%%%%%%%%%%%%%%%
 \newpage
\begin{itemize}

\item Adults who were 
widowed at the time of the interview, had the highest diabetes prevalence among marital status. Refer to figure \ref{fig:marital.Diabetes.2013}.

\item Those who were separated had 53\% less possibility of having diabetes prevalence when compared with those who responded being married. This difference was significant (p-value $<$ 0.05). Refere to Table \ref{tab:marital.Diabetes.2013}.

\end{itemize}

\begin{figure}[H]
\caption{Diabetes prevalence among adults by marital status,
         2013}
\label{fig:marital.Diabetes.2013}
\begin{knitrout}
\definecolor{shadecolor}{rgb}{0.969, 0.969, 0.969}\color{fgcolor}

{\centering \includegraphics[width=\maxwidth]{/media/truecrypt2/ORP2/BRFSS/Objects/Country/Puerto_Rico/Disease/Diabetes/Prevalence/Adult/2013marital2-1} 

}



\end{knitrout}
 \end{figure}

%latex.default(tabl.marital, title = "Variables", file = "", append = TRUE,     rgroup = "Marital group", colhead = c("Prevalence", "Cases (N)",         "OR", "OR(SE)", "p-value"), longtable = F, table.env = T,     here = T, caption = paste(set_Measure, " among adults by marital status,",         set_Years, sep = " "), label = paste("tab:marital", set_DiseaseF,         set_Years, sep = "."))%
\begin{table}[H]
\caption{Diabetes prevalence  among adults by marital status, 2013\label{tab:marital.Diabetes.2013}} 
\begin{center}
\begin{tabular}{llllll}
\hline\hline
\multicolumn{1}{l}{Variables}&\multicolumn{1}{c}{Prevalence}&\multicolumn{1}{c}{Cases (N)}&\multicolumn{1}{c}{OR}&\multicolumn{1}{c}{OR(SE)}&\multicolumn{1}{c}{p-value}\tabularnewline
\hline
{\bfseries Marital group}&&&&&\tabularnewline
~~Married&18.0 (16.3-19.7)&195,657&1.00&0.00&1.00\tabularnewline
~~Divorced&14.6 (11.5-17.6)& 76,977&0.76&0.16&0.11\tabularnewline
~~Widowed&29.7 (25.6-33.9)& 88,654&0.76&0.15&0.08\tabularnewline
~~Separated&9.01 (5.68-12.3)& 21,446&0.47&0.24&0.00\tabularnewline
~~Never Married&4.66 (3.32-6.00)& 22,882&1.19&0.20&0.39\tabularnewline
~~Unmarried Couple&7.63 (5.15-10.1)& 12,613&0.90&0.22&0.64\tabularnewline
\hline
\end{tabular}\end{center}

\end{table}

 
%%%%%%%%%%%%%%%%%% employment status %%%%%%%%%%
 \newpage
\begin{itemize}


\item Those adults who were 
retired at the moment of the interview, had the highest diabetes prevalence (Figure \ref{fig:employ.Diabetes.2013}).

\item Adults who work at home, retired, or unable to work had 1.97 times more possibility, 2.06 times more possibility, and 2.27 times more possibility, respectively, of reporting diabetes prevalence when compared with those who responded being employed. The differences were significant (p-value $<$ 0.05). 
Data shown in Table \ref{tab:employ.Diabetes.2013}.

\end{itemize}

\begin{figure}[H]
\caption{Diabetes prevalence among adults by employment status, 
         2013}
\label{fig:employ.Diabetes.2013}
\begin{knitrout}
\definecolor{shadecolor}{rgb}{0.969, 0.969, 0.969}\color{fgcolor}

{\centering \includegraphics[width=\maxwidth]{/media/truecrypt2/ORP2/BRFSS/Objects/Country/Puerto_Rico/Disease/Diabetes/Prevalence/Adult/2013employ-1} 

}



\end{knitrout}
 \end{figure}

%latex.default(tabl.emplrec2, title = "Variables", file = "",     append = TRUE, rgroup = "Employment status", colhead = c("Prevalence",         "Cases (N)", "OR", "OR(SE)", "p-value"), longtable = F,     table.env = T, here = T, caption = paste(set_Measure, " among adults by employment status,",         set_Years, sep = " "), label = paste("tab:employ", set_DiseaseF,         set_Years, sep = "."))%
\begin{table}[H]
\caption{Diabetes prevalence  among adults by employment status, 2013\label{tab:employ.Diabetes.2013}} 
\begin{center}
\begin{tabular}{llllll}
\hline\hline
\multicolumn{1}{l}{Variables}&\multicolumn{1}{c}{Prevalence}&\multicolumn{1}{c}{Cases (N)}&\multicolumn{1}{c}{OR}&\multicolumn{1}{c}{OR(SE)}&\multicolumn{1}{c}{p-value}\tabularnewline
\hline
{\bfseries Employment status}&&&&&\tabularnewline
~~Employed&6.24 (4.99-7.50)&  3,710&1.00&0.00&1.00\tabularnewline
~~Out of work&7.62 (4.63-10.6)& 12,056&1.24&0.27&0.41\tabularnewline
~~Homework&20.6 (17.7-23.4)& 40,322&1.97&0.19&0.00\tabularnewline
~~Student&0.72 (0.00-1.84)& 65,561&0.64&0.84&0.60\tabularnewline
~~Retired&33.0 (30.0-36.0)&102,585&2.06&0.18&0.00\tabularnewline
~~Unable to work&26.3 (21.5-31.0)&193,995&2.27&0.20&0.00\tabularnewline
\hline
\end{tabular}\end{center}

\end{table}


%%%%%%%%%%%%%%%% HRQOL %%%%%%%%%%%%%%%%%%%%%%%%%
 \newpage
\subsubsection{Health related quality of life}


 \begin{itemize}

\item Figure \ref{fig:fairpoor.Diabetes.2013} shows that diabetes prevalence  was 
\ifthenelse{
 706 < 
  2901}{higher}{lower} in persons who perceive their health has fair / poor,
than those who claim to have a very good health.

\item Adults who perceive their health as fair or poor had 11\% less possibility of reporting diabetes prevalence when compared with those who perceived their health as very good. This difference was not significant (p-value $>$ 0.05). Data shown in Table \ref{tab:fairpoor.Diabetes.2013}.

\end{itemize}

\begin{figure}[H]
\caption{Diabetes prevalence among adults according to health perception,
         2013}
\label{fig:fairpoor.Diabetes.2013}

\begin{knitrout}
\definecolor{shadecolor}{rgb}{0.969, 0.969, 0.969}\color{fgcolor}

{\centering \includegraphics[width=\maxwidth]{/media/truecrypt2/ORP2/BRFSS/Objects/Country/Puerto_Rico/Disease/Diabetes/Prevalence/Adult/2013poorh-1} 

}



\end{knitrout}
 \end{figure}

%latex.default(tabl.fairpoor, title = "Variables", file = "",     append = TRUE, rgroup = "Health Perception", colhead = c("Prevalence",         "Cases (N)", "OR", "OR(SE)", "p-value"), longtable = F,     table.env = T, here = T, caption = paste(set_Measure, " among adults according to health perception,",         set_Years, sep = " "), label = paste("tab:fairpoor",         set_DiseaseF, set_Years, sep = "."))%
\begin{table}[H]
\caption{Diabetes prevalence  among adults according to health perception, 2013\label{tab:fairpoor.Diabetes.2013}} 
\begin{center}
\begin{tabular}{llllll}
\hline\hline
\multicolumn{1}{l}{Variables}&\multicolumn{1}{c}{Prevalence}&\multicolumn{1}{c}{Cases (N)}&\multicolumn{1}{c}{OR}&\multicolumn{1}{c}{OR(SE)}&\multicolumn{1}{c}{p-value}\tabularnewline
\hline
{\bfseries Health Perception}&&&&&\tabularnewline
~~Very Good&7.06 (6.09-8.03)&128,057&1.00&0.00&1.00\tabularnewline
~~Fair / Poor&29.0 (26.7-31.3)&288,091&0.89&0.22&0.63\tabularnewline
\hline
\end{tabular}\end{center}

\end{table}

 
\newpage

%%%%%%%%%%%%% physical unhealthy %%%%%%%%%%%%
 \newpage
\begin{itemize}

\item Adults who reported being physically unhealthy for 14 days or more in the past 30 days, had 
\ifthenelse{
 1284 >
  2654}{lower}{higher} diabetes prevalence when compared with 
those who reported being physically unhealthy for less than 13 days in the last 30 days.

%\item Table \ref{tab:phys} shows the estimations of the set_Measure.

\item Adults who felt physically impaired for 14 days or more had a non-significant 8\% less possibility of reporting having diabetes at the moment of the interview, when compared with those who felt physically impaired for 13 days or less (See Table \ref{tab:phys.Diabetes.2013}).

\end{itemize}

\begin{figure}[H]
\caption{Diabetes prevalence among adults according to physical health perception, 2013}
\label{fig:phys.Diabetes.2013}
\begin{knitrout}
\definecolor{shadecolor}{rgb}{0.969, 0.969, 0.969}\color{fgcolor}

{\centering \includegraphics[width=\maxwidth]{/media/truecrypt2/ORP2/BRFSS/Objects/Country/Puerto_Rico/Disease/Diabetes/Prevalence/Adult/2013physchlcat-1} 

}



\end{knitrout}
 \end{figure}

%latex.default(tabl.physhlcat, title = "Variables", file = "",     append = TRUE, rgroup = "Days with physical symptoms", colhead = c("Prevalence",         "Cases (N)", "OR", "OR(SE)", "p-value"), longtable = F,     table.env = T, here = T, caption = paste(set_Measure, "among adults by days with physical symptoms,",         set_Years, sep = " "), label = paste("tab:phys", set_DiseaseF,         set_Years, sep = "."))%
\begin{table}[H]
\caption{Diabetes prevalence among adults by days with physical symptoms, 2013\label{tab:phys.Diabetes.2013}} 
\begin{center}
\begin{tabular}{llllll}
\hline\hline
\multicolumn{1}{l}{Variables}&\multicolumn{1}{c}{Prevalence}&\multicolumn{1}{c}{Cases (N)}&\multicolumn{1}{c}{OR}&\multicolumn{1}{c}{OR(SE)}&\multicolumn{1}{c}{p-value}\tabularnewline
\hline
{\bfseries Days with physical symptoms}&&&&&\tabularnewline
~~0-13 days&12.8 (11.7-13.9)&306,908&1.00&0.00&1.00\tabularnewline
~~14 or more days&26.5 (23.0-30.0)&108,843&0.92&0.22&0.72\tabularnewline
\hline
\end{tabular}\end{center}

\end{table}


%%%%%%%%%%%%%%%%%%%%%%%% Mentally unhelathy %%%%%%%%%%%%%%
 \newpage
\begin{itemize}

\item Diabetes prevalence among adults who reported feeling mentally unhealthy for more than 13 days in the last month was non-significantlly \ifthenelse{1435 > 
  1803}{lower}{higher} when compared with their counterpart (Figure \ref{fig:mental.Diabetes.2013}).

%\item set_Measure for this groups are presented in the Table \ref{tab:mental}.

\item  Adults who felt mentally impaired for 14 or more days had 40\% less possibility of having diabetes prevalence when compared with those who answered 0-13 days. This difference was not significant (p-value $>$ 0.05). (See Table \ref{tab:mental.Diabetes.2013}).

\end{itemize}

\begin{figure}[H]
\centering
\caption{Diabetes prevalence among adults by days of frequent mental distress, 2013}
\label{fig:mental.Diabetes.2013}

\begin{knitrout}
\definecolor{shadecolor}{rgb}{0.969, 0.969, 0.969}\color{fgcolor}

{\centering \includegraphics[width=\maxwidth]{/media/truecrypt2/ORP2/BRFSS/Objects/Country/Puerto_Rico/Disease/Diabetes/Prevalence/Adult/2013mentdist-1} 

}



\end{knitrout}
 \end{figure}

%latex.default(tabl.mentdist, title = "Variables", file = "",     append = TRUE, rgroup = "Mentally Unhealthy", colhead = c("Prevalence",         "Cases (N)", "OR", "OR(SE)", "p-value"), longtable = F,     table.env = T, here = T, caption = paste(set_Measure, "among",         set_Population, "by days of frequent mental distress,",         set_Years, sep = " "), label = paste("tab:mental", set_DiseaseF,         set_Years, sep = "."))%
\begin{table}[H]
\caption{Diabetes prevalence among adults by days of frequent mental distress, 2013\label{tab:mental.Diabetes.2013}} 
\begin{center}
\begin{tabular}{llllll}
\hline\hline
\multicolumn{1}{l}{Variables}&\multicolumn{1}{c}{Prevalence}&\multicolumn{1}{c}{Cases (N)}&\multicolumn{1}{c}{OR}&\multicolumn{1}{c}{OR(SE)}&\multicolumn{1}{c}{p-value}\tabularnewline
\hline
{\bfseries Mentally Unhealthy}&&&&&\tabularnewline
~~0-13 days&14.3 (13.2-15.4)&344,215&1.00&0.00&1.00\tabularnewline
~~14 or more days&18.0 (14.7-21.2)& 69,979&0.60&0.28&0.08\tabularnewline
\hline
\end{tabular}\end{center}

\end{table}



%%%%%%%%%%%%%%%%%%% poorhlcat %%%%%%%%%%%%%%%%%%%%%%%%%%%%%
 \newpage
\begin{itemize}

\item A \ifthenelse{
 1490 > 
  2399}{lower}{higher} 
diabetes prevalence was observed among adults who were unable to perform their usual activities for 14 days or more due to physical or mental impairment, when compared with the other group. Refer to Figure \ref{fig:poor.Diabetes.2013}.



\item Persons who could not conduct their usual daily activities for more than 14 days in the last month had 6\% less possibility of reporting diabetes when compared with those who felt this way for at most 13 days in the last 30. The difference was not statistically significant (p $<$ 0.05).

\end{itemize}

\begin{figure}[H]
\caption{Diabetes prevalence among adults who were unable to conduct usual activities due to physical or mental impediment, 2013}
\label{fig:poor.Diabetes.2013}

\begin{knitrout}
\definecolor{shadecolor}{rgb}{0.969, 0.969, 0.969}\color{fgcolor}

{\centering \includegraphics[width=\maxwidth]{/media/truecrypt2/ORP2/BRFSS/Objects/Country/Puerto_Rico/Disease/Diabetes/Prevalence/Adult/2013poor-1} 

}


\end{knitrout}
\end{figure}

%latex.default(tabl.poorhlcat, title = "Variables", file = "",     append = TRUE, rgroup = "Activiy limitation", colhead = c("Prevalence",         "Cases (N)", "OR", "OR(SE)", "p-value"), longtable = F,     table.env = T, here = T, caption = paste(set_Measure, "among",         set_Population, "who were unable to cunduct usual activities due to physical or mental impairment,",         set_Years, sep = " "), label = paste("tab:poor", set_DiseaseF,         set_Years, sep = "."))%
\begin{table}[H]
\caption{Diabetes prevalence among adults who were unable to cunduct usual activities due to physical or mental impairment, 2013\label{tab:poor.Diabetes.2013}} 
\begin{center}
\begin{tabular}{llllll}
\hline\hline
\multicolumn{1}{l}{Variables}&\multicolumn{1}{c}{Prevalence}&\multicolumn{1}{c}{Cases (N)}&\multicolumn{1}{c}{OR}&\multicolumn{1}{c}{OR(SE)}&\multicolumn{1}{c}{p-value}\tabularnewline
\hline
{\bfseries Activiy limitation}&&&&&\tabularnewline
~~0-13 days&14.9 (13.1-16.6)&147,138&1.00&0.00&1.00\tabularnewline
~~14 or more days&23.9 (19.6-28.3)& 59,389&0.94&0.30&0.84\tabularnewline
\hline
\end{tabular}\end{center}

\end{table}


%%%%%%%%%%%%%%%%%%%%% unhlthy %%%%%%%%%%%%%%%%%%%%%%%%%%
\newpage
\begin{itemize}

\item As seen in Figure \ref{fig:unhlthy.Diabetes.2013}, those adults who felt mentally or physically unhealthy for more than 14 days in the last 30 days, had a significant \ifthenelse{
13>
20}{lower}{higher} 
diabetes prevalence when compared with their counterpart.

%\item The prevalence for those who felt mentally or fisically unhealthy is more than twice of the
%prevalence of those who felt healthy for at least 14 days in the past 30(Table \ref{tab:unhlthy}).

\item Those persons who felt mentally or physically unhealthy for more than 14 days in the last month had 22\% less possibility of reporting diabetes when compared with those who felt mentally or physically unhealthy for at most 13 days in the last 30 days. Data shown in Table \ref{tab:unhlthy.Diabetes.2013}.

\end{itemize}

\begin{figure}[H]
\caption{Diabetes prevalence among adults by unhealthy days, 2013}
\label{fig:unhlthy.Diabetes.2013}

\begin{knitrout}
\definecolor{shadecolor}{rgb}{0.969, 0.969, 0.969}\color{fgcolor}

{\centering \includegraphics[width=\maxwidth]{/media/truecrypt2/ORP2/BRFSS/Objects/Country/Puerto_Rico/Disease/Diabetes/Prevalence/Adult/2013unhealthy-1} 

}



\end{knitrout}
\end{figure}

%latex.default(tabl.unhlthy, title = "Variables", file = "", append = TRUE,     rgroup = "Unhealthy Days", colhead = c("Prevalence", "Cases (N)",         "OR", "OR(SE)", "p-value"), longtable = F, table.env = T,     here = T, caption = paste(set_Measure, "among", set_Population,         "by unhealthy days,", set_Years, sep = " "), label = paste("tab:unhlthy",         set_DiseaseF, set_Years, sep = "."))%
\begin{table}[H]
\caption{Diabetes prevalence among adults by unhealthy days, 2013\label{tab:unhlthy.Diabetes.2013}} 
\begin{center}
\begin{tabular}{llllll}
\hline\hline
\multicolumn{1}{l}{Variables}&\multicolumn{1}{c}{Prevalence}&\multicolumn{1}{c}{Cases (N)}&\multicolumn{1}{c}{OR}&\multicolumn{1}{c}{OR(SE)}&\multicolumn{1}{c}{p-value}\tabularnewline
\hline
{\bfseries Unhealthy Days}&&&&&\tabularnewline
~~0-13 days&13.0 (11.8-14.1)&279,489&1.00&0.00&1.00\tabularnewline
~~14 or more days&20.9 (18.4-23.4)&138,740&0.78&0.22&0.27\tabularnewline
\hline
\end{tabular}\end{center}

\end{table}

\newpage
 \subsubsection{Interpretation}

An estimate of 418,229 
(14.88\%) adult in Puerto Rico had diabetes as of the period of 2013.
When evaluating diabetes by age groups we observe a strong incrase in diabetes prevalence with an incrase in age group. The 65+ age group have 20 times more possibility of reporting diabetes when compared with the 18-24 age group. In estimate, about 200,000 indiviudals age 65 or more reported having the condition. With regards to gender and diabetes there was not a statistical nor meaningfull differences in prevalence or in risk.  Those in the education group with some university or university graduates had significantly less possibility of reporting being diagnose with diabetes when compared with the Some High School group.  Although marginally significant those in the household income range of 35k-50k and those in the range of more than 50k annually, had less possibility of reporting being diagnose with diabetes than those in the group of less than 15k annually. In addition, was observed a constant decrease in prevalence form the less to most affluent group. In marital status, the group that reported being separated had significant less risk  of reporting diabetes than the Married group. Althought the Widowed group had the higer prevalence the one who had the higher risk was the never married group with the lowest prevalence. Nevertheless, the risk was not significant.

Moving to the health related quality of life measures, the possibility of reporting diabetes prevalence in adults that perceived their health as fair or poor was 0.11 times lower than those who considered their health as good, very good or excellent. The adults who said that they felt physically bad for 14 days or more for the last 30 days had 0.08 times lower possibility of reporting diabetes prevalence than those less than 13 days. In the same manner, adults who felt mentally distressed for 14 days or more in a month had 0.4 times lower possibility of reporting diabetes prevalence than those less than 13 days feeling mentally distressed. The adults who were unable to carry on with their normal activities more than 14 days in a month had 0.06 less risk than the comparison group. Non of the health related quality of life comparisons were statistically diferent. The results of HRQOL on diabetes respondant suggest this population does not perceived that this condition increase the risk of impair their quality of life.


%Summary tables
\newpage
\subsubsection{Summary tables}
In this section we provide a set of summary tables of diabetes among adults Puerto Rico for the years 2013. The tables aggregate all the analysis conducted in this report for an easy to print and carry document for reference, or to be used as an annex for the preparation of other documents such as an application for funds. For the interpretation of each result, please refer to the corresponding section in the document.

% Overall
\begin{table}[H]
\caption{Diabetes prevalence among adults in Puerto Rico, 2013\label{tab:Overall.tabl.Diabetes.2013}} 
\begin{center}
\begin{tabular}{llll}
\hline\hline
\multicolumn{1}{l}{Group}&\multicolumn{1}{c}{Prevalence}&\multicolumn{1}{c}{Cases}&\multicolumn{1}{c}{Sample Size}\tabularnewline
\hline
Adults&14.88 (13.81-15.95)&418,229&6,011\tabularnewline
\hline
\end{tabular}\end{center}

\end{table}


%latex.default(object = SocioD.tabl, title = "Variables", file = "",     append = TRUE, rgroup = c("Age group", "Gender", "Escolarity",         "Household income", "Marital status", "Employment status"),     n.rgroup = c(6, 2, 4, 5, 6, 6), table.env = T, longtable = F,     here = T, caption = paste(set_Measure, " among adults by socio-demographic variables,",         set_Years, sep = " "), label = "tab:SocioD.tabl")%
\begin{table}[H]
\caption{Diabetes prevalence  among adults by socio-demographic variables, 2013\label{tab:SocioD.tabl}} 
\begin{center}
\begin{tabular}{llllll}
\hline\hline
\multicolumn{1}{l}{Variables}&\multicolumn{1}{c}{Prevalence}&\multicolumn{1}{c}{Number}&\multicolumn{1}{c}{OR}&\multicolumn{1}{c}{OR(SE)}&\multicolumn{1}{c}{p-value}\tabularnewline
\hline
{\bfseries Age group}&&&&&\tabularnewline
~~18-24&0.98 (0.10-1.86)&  3,710&1.00&0.00&1.00\tabularnewline
~~25-34&2.47 (0.99-3.96)& 12,056&1.69&0.55&0.34\tabularnewline
~~35-44&8.36 (5.63-11.0)& 40,322&5.94&0.50&0.00\tabularnewline
~~45-54&13.5 (10.8-16.2)& 65,561&9.49&0.49&0.00\tabularnewline
~~55-64&23.5 (20.4-26.6)&102,585&14.9&0.49&0.00\tabularnewline
~~65+&35.5 (32.8-38.1)&193,995&20.3&0.49&0.00\tabularnewline
\hline
{\bfseries Gender}&&&&&\tabularnewline
~~Males&13.9 (12.3-15.6)&183,567&1.00&0.00&1.00\tabularnewline
~~Females&15.6 (14.3-17.0)&234,661&0.97&0.11&0.82\tabularnewline
\hline
{\bfseries Escolarity}&&&&&\tabularnewline
~~Some High School&25.4 (22.7-28.2)&202,434&1.00&0.00&1.00\tabularnewline
~~High School Graduate&14.5 (12.5-16.6)&105,865&0.94&0.14&0.67\tabularnewline
~~Some University&8.92 (7.38-10.4)& 63,386&0.76&0.15&0.08\tabularnewline
~~University Graduate&8.02 (6.73-9.31)& 45,863&0.63&0.17&0.01\tabularnewline
\hline
{\bfseries Household income}&&&&&\tabularnewline
~~\textless15k&18.5 (16.7-20.4)&223,365&1.00&0.00&1.00\tabularnewline
~~15k-\textless25k&12.1 (10.0-14.2)& 78,437&0.84&0.13&0.23\tabularnewline
~~25k-\textless35k&11.3 (8.44-14.3)& 22,934&0.90&0.19&0.60\tabularnewline
~~35k-\textless50k&8.63 (5.41-11.8)& 12,497&0.87&0.26&0.61\tabularnewline
~~50+k&6.28 (4.10-8.46)&  8,948&0.66&0.23&0.08\tabularnewline
\hline
{\bfseries Marital status}&&&&&\tabularnewline
~~Married&18.0 (16.3-19.7)&195,657&1.00&0.00&1.00\tabularnewline
~~Divorced&14.6 (11.5-17.6)& 76,977&0.76&0.16&0.11\tabularnewline
~~Widowed&29.7 (25.6-33.9)& 88,654&0.76&0.15&0.08\tabularnewline
~~Separated&9.01 (5.68-12.3)& 21,446&0.47&0.24&0.00\tabularnewline
~~Never Married&4.66 (3.32-6.00)& 22,882&1.19&0.20&0.39\tabularnewline
~~Unmarried Couple&7.63 (5.15-10.1)& 12,613&0.90&0.22&0.64\tabularnewline
\hline
{\bfseries Employment status}&&&&&\tabularnewline
~~Employed&6.24 (4.99-7.50)&  3,710&1.00&0.00&1.00\tabularnewline
~~Out of work&7.62 (4.63-10.6)& 12,056&1.24&0.27&0.41\tabularnewline
~~Homework&20.6 (17.7-23.4)& 40,322&1.97&0.19&0.00\tabularnewline
~~Student&0.72 (0.00-1.84)& 65,561&0.64&0.84&0.60\tabularnewline
~~Retired&33.0 (30.0-36.0)&102,585&2.06&0.18&0.00\tabularnewline
~~Unable to work&26.3 (21.5-31.0)&193,995&2.27&0.20&0.00\tabularnewline
\hline
\end{tabular}\end{center}

\end{table}

 

 
%latex.default(object = hrqol.tabl, title = "Variables", file = "",     append = TRUE, rgroup = c("Health perception", "Physical unhealthy",         "Mental unhealthy", "Activity limitation", "Physical and mental unhealthy"),     n.rgroup = c(2, 2, 2, 2, 2), rowname = c("Very Good", "Fair / Poor",         rep(c("0-13 days", "14 or more days"), 4)), table.env = T,     here = T, caption = paste(set_Measure, " among adults by health related quality of life variables,",         set_Years, sep = " "), label = "tab:hrqol.tabl")%
\begin{table}[H]
\caption{Diabetes prevalence  among adults by health related quality of life variables, 2013\label{tab:hrqol.tabl}} 
\begin{center}
\begin{tabular}{llllll}
\hline\hline
\multicolumn{1}{l}{Variables}&\multicolumn{1}{c}{Prevalence}&\multicolumn{1}{c}{Number}&\multicolumn{1}{c}{OR}&\multicolumn{1}{c}{OR(SE)}&\multicolumn{1}{c}{p-value}\tabularnewline
\hline
{\bfseries Health perception}&&&&&\tabularnewline
~~Very Good&7.06 (6.09-8.03)&128,057&1.00&0.00&1.00\tabularnewline
~~Fair / Poor&29.0 (26.7-31.3)&288,091&0.89&0.22&0.63\tabularnewline
\hline
{\bfseries Physical unhealthy}&&&&&\tabularnewline
~~0-13 days&12.8 (11.7-13.9)&306,908&1.00&0.00&1.00\tabularnewline
~~14 or more days&26.5 (23.0-30.0)&108,843&0.92&0.22&0.72\tabularnewline
\hline
{\bfseries Mental unhealthy}&&&&&\tabularnewline
~~0-13 days&14.3 (13.2-15.4)&344,215&1.00&0.00&1.00\tabularnewline
~~14 or more days&18.0 (14.7-21.2)& 69,979&0.60&0.28&0.08\tabularnewline
\hline
{\bfseries Activity limitation}&&&&&\tabularnewline
~~0-13 days&14.9 (13.1-16.6)&147,138&1.00&0.00&1.00\tabularnewline
~~14 or more days&23.9 (19.6-28.3)& 59,389&0.94&0.30&0.84\tabularnewline
\hline
{\bfseries Physical and mental unhealthy}&&&&&\tabularnewline
~~0-13 days&13.0 (11.8-14.1)&279,489&1.00&0.00&1.00\tabularnewline
~~14 or more days&20.9 (18.4-23.4)&138,740&0.78&0.22&0.27\tabularnewline
\hline
\end{tabular}\end{center}

\end{table}

 
