\section*{Bibliography}\label{bibliography}
\addcontentsline{toc}{section}{Bibliography}

Abegunde, D. O., Mathers, C. D., Adam, T., Ortegon, M., \& Strong, K.
(2007). The burden and costs of chronic diseases in low-income and
middle-income countries. \emph{The Lancet}, \emph{370}(9603),
1929--1938.

Alwan, A., MacLean, D. R., Riley, L. M., d'Espaignet, E. T., Mathers, C.
D., Stevens, G. A., \& Bettcher, D. (2010). Monitoring and surveillance
of chronic non-communicable diseases: Progress and capacity in
high-burden countries. \emph{The Lancet}, \emph{376}(9755), 1861--1868.

Anderson, G. F. (2010). \emph{Chronic care: Making the case for ongoing
care}. Robert Wood Johnson Foundation.

Assembly, U. G. (2011). Political declaration of the high-level meeting
of the general assembly on the prevention and control of
non-communicable diseases. \emph{UN New York}.

Bloom, D. E., Cafiero, E., Jan{é}-Llopis, E., Abrahams-Gessel, S.,
Bloom, L. R., Fathima, S., \ldots{} others. (2012). \emph{The global
economic burden of noncommunicable diseases}. Program on the Global
Demography of Aging.

Daar, A. S., Singer, P. A., Persad, D. L., Pramming, S. K., Matthews, D.
R., Beaglehole, R., \ldots{} others. (2007). Grand challenges in chronic
non-communicable diseases. \emph{Nature}, \emph{450}(7169), 494--496.

Ford, E. S., \& Capewell, S. (2011). Proportion of the decline in
cardiovascular mortality disease due to prevention versus treatment:
Public health versus clinical care. \emph{Annual Review of Public
Health}, \emph{32}, 5--22.

Hunter, D. J., \& Reddy, K. S. (2013). Noncommunicable diseases.
\emph{New England Journal of Medicine}, \emph{369}(14), 1336--1343.

Jamison, D. T., Breman, J. G., Measham, A. R., Alleyne, G., Claeson, M.,
Evans, D. B., \ldots{} Musgrove, P. (2006). \emph{Disease control
priorities in developing countries}. World Bank Publications.

Murray, C. J., Vos, T., Lozano, R., Naghavi, M., Flaxman, A. D.,
Michaud, C., \ldots{} others. (2013). Disability-adjusted life years
(dALYs) for 291 diseases and injuries in 21 regions, 1990--2010: A
systematic analysis for the global burden of disease study 2010.
\emph{The Lancet}, \emph{380}(9859), 2197--2223.

Nikolic, I. A., Stanciole, A. E., \& Zaydman, M. (2011). Chronic
emergency: Why nCDs matter.

Porta, M., Greenland, S., Hern{á}n, M., Silva, I. D. S., \& Last, J. M.
(2014). \emph{A dictionary of epidemiology}. Oxford University Press.

Remington, P. L., Brownson, R. C., \& Wegner, M. V. (2010). Chronic
disease epidemiology and control.

Riley, L., \& Cowan, M. (2014). Noncommunicable diseases country
profiles 2014. Retrieved from
\url{http://www.who.int/global-coordination-mechanism/publications/ncds-country-profiles-eng.pdf}

Sytkowski, P. A., Kannel, W. B., \& D'Agostino, R. B. (1990). Changes in
risk factors and the decline in mortality from cardiovascular disease:
The framingham heart study. \emph{New England Journal of Medicine},
\emph{322}(23), 1635--1641.

The american heritage® science dictionary. (2015). Retrieved from
\href{http://dictionary.reference.com/browse/heart attack}{http://dictionary.reference.com/browse/heart attack}

Ward, B. W., Schiller, J. S., \& Goodman, R. A. (2014). Peer reviewed:
Multiple chronic conditions among uS adults: A 2012 update.
\emph{Preventing Chronic Disease}, \emph{11}.

WHO. (2013). Draft comprehensive global monitoring framework and targets
for the prevention and control of noncommunicable diseases. Retrieved
from \url{http://apps.who.int/gb/ebwha/pdf_files/WHA66/A66_8-en.pdf}

WHO, \& others. (2005). WHO sTEPS surveillance manual: The wHO sTEPwise
approach to chronic disease risk factor surveillance.

© encyclopedia britannica, inc. (2015). Retrieved from
\href{http://dictionary.reference.com/browse/cardiovascular disease}{http://dictionary.reference.com/browse/cardiovascular disease}
